\documentclass[11pt,a4paper]{scrartcl}
\usepackage[ngerman]{babel}                         % Word wrap for German new ortography
\usepackage[utf8]{inputenc}                         % Interpret the document as UTF-8 (Proper handling of some special characters)
\usepackage[T1]{fontenc}                            % Use 8-bit font encoding (for "real" umlauts support)
\usepackage{multicol}                               % Multiple colums of text for better readability
\usepackage{amsmath}                                %
\usepackage{amssymb}                                %
\usepackage{amsthm}                                 %
\usepackage{marvosym}                               % Lots of symbols
\usepackage{graphicx}                               % Include images
\usepackage{hyperref}                               % Correct handling of URL's (wrapping, link in the resulting PDF)
\usepackage{lipsum}                                 % Blindtext
\usepackage{float}                                  % Better image placement
\usepackage{listings}                               % Source code listings
\usepackage{color}
\definecolor{darkgreen}{rgb}{0,.5,0}
\definecolor{darkred}{rgb}{.5,0,0}

\lstset{
  frame=single,
  basicstyle=\ttfamily,
  showstringspaces=false,
  commentstyle=\itshape\color{darkgreen},
  morekeywords={func,guard,let,if,else,return},
  keywordstyle=\color{blue},
  stringstyle=\color{darkred},
  breaklines=true,
  captionpos=b
}
\usepackage{scrlayer-scrpage}                       % Replacement for fancyhdr with KOMA
\usepackage{csquotes}                               % BibLaTeX wants this
\usepackage[
  backend=bibtex8,
  sorting=none
]{biblatex}                                         % Replacement for BibTeX with @online sources (among others)

\title{SSL Downgrade von IOT-Geräten}
\subtitle{Seminar ``Das Internet der Dinge -- Ein Hackerparadies?''}
\author{Alexander Korff\and
Sergej Maul\and
Yannik Stöcklin\and
Sebastian Philipp\and
Daniel Seidinger\and
Fabian Neumeier\and
Lukas Stöckli\and
Manuel Rickli\and
Désirée Nusch\and
Samuel Hugger\and
Clement Francois\and
Joel Grossenbacher}
\ihead{SSL Downgrade von IOT-Geräten}

\addbibresource{ssldowngrade-report.bib}

\begin{document}
\maketitle
\section{Lorem ipsum}
\lipsum
\section{Lokale Kommunikation mit der Philips Hue}
Nach einigen Beobachtungen stellten wir fest, dass in der Zeit zwischen dem Befehl an das
Amazon Echo, das Licht anzuschalten, und dem tatsächlichen Aufleuchten der Lampe kein
Traffic zwischen der Philips Hue Bridge und dem Internet zu beobachten war. Daher nahmen wir
an, dass der Befehl zum Anschalten im lokalen Netzwerk gesendet wird. Lokaler Traffic würde
auch nicht in den mitgeschnittenen Paketen enthalten sein, da wir ARP Spoofing nur zwischen
der FRITZ!Box und der Hue Bridge betrieben hatten.

Entsprechend änderten wir unser Vorgehen,
um den Traffic zwischen dem Echo und der Hue Bridge mitzuschneiden. Dies war sofort
erfolgreich; wir konnten feststellen, dass das Echo eine HTTP Request im Klartext an die Hue
Bridge sendet. Eine auf das Wesentliche gekürzte Version eines solchen Requests zum
Ausschalten einer Lampe ist in Listing~\ref{lst:http-hueoff} dargestellt.
\begin{lstlisting}[label={lst:http-hueoff},caption={Ein HTTP Request zum Ausschalten von Lampe Nr. 2.},float,floatplacement=H]
PUT /api/-UaDNHD5j44y07zYdAOEg0JuIakQpu72ivJXXHVS/lights/2/state HTTP/1.1
Content-Type: application/x-www-form-urlencoded
Content-Length: 13

{"on": false}
\end{lstlisting}

Abgesehen davon, dass der falsche Wert für \path{Content-Type} angegeben wird (anstatt dem
tatsächlichen \path{application/json}), fielen folgende Punkte auf:
\begin{itemize}
\item Die Request enthält eine nicht weiter bekannte Art von Zugriffsberechtigung oder ID in
der URL. Dies stellte jedoch kein Hindernis dar; die ID liess sich beliebig oft
wiederverwenden.
\item Die Lampe, die angesprochen wird, wird ebenfalls numerisch in der URL übergeben, in
diesem Fall Lampe Nr. 2.
\item All dies geschieht komplett unverschlüsselt; ein SSL Downgrade war daher gar nicht
notwendig.
\end{itemize}
Analog konnten wir die Befehle zum Ausschalten, zum Setzen der Helligkeit sowie der Farbe
mitschneiden und auch problemlos selbst versenden. Da sich die Farbe mit dem Amazon Echo nur
verändern lässt, wenn man einen Philips Hue Account verwendet, schnitten wir hierzu den
Datenverkehr zwischen der Hue Bridge und einem Smartphone mit, auf welchem die zugehörige
Software verwendet wurde, welche das Setzen der Farbe auch ohne Login erlaubt. Auch hier war
wieder keine Verschlüsselung im Spiel.Mit dem HTTP Request in Listing~\ref{lst:http-huecol}
wird die Lampe angeschalten und auf eine gegebene Helligkeit und Farbe gesetzt. Die
Helligkeit ist eine einzelne Zahl im Bereich zwischen $0$ und $255$. Die Farbe wird durch
zwei Werte ``x'' und ``y'' zwischen $0$ und $1$ beschrieben; die Bedeutung dieser Werte
erschloss sich uns jedoch nicht. Wir stellten hierzu zwei Hypothesen auf:
\begin{itemize}
\item Die Werte ensprechen den (normalisierten) Koordinaten der Farbe im Farbauswahldialog
der Smartphone-App.
\item Die Werte entsprechen dem Farbton und der Sättigung im HSV-Farbmodell; die
Helligkeitskomponente wird separat übergeben.
\end{itemize}
Beide Hypothesen liessen sich durch Versuche falsifizieren. Die API ist zwar vom Hersteller
dokumentiert, aber erfordern wiederum einen Login, um darauf zuzugreifen.
\cite{philips:developer} Aus Zeitgründen wurde diese API nicht weiter untersucht.
\begin{lstlisting}[label={lst:http-huecol},caption={Ein HTTP Request für bunte Lampen.},float,floatplacement=H]
PUT /api/-UaDNHD5j44y07zYdAOEg0JuIakQpu72ivJXXHVS/lights/2/state HTTP/1.1
Content-Type: application/json
Content-Length: 42

{"on": true, "bri": 255, "xy": [1.0, 1.0]}
\end{lstlisting}

\section{LED Matrix}
Die in einem früheren Projekt der Vorlesung \textit{Computer Architecture and Operating Systems} gebaute LED Matrix wurde im Verlauf des Seminars bereits um einige Funktionalitäten im Bereich des Internets der Dinge erweitert. Unter anderem ist es möglich mittels Sprachbefehlen (Amazon Alexa) das aktuelle Wetter, eine Notiz oder auch verschiedene Farben anzuzeigen zu lassen. Die Matrix wird mit einem Arduino Mega angesteuert, welcher seine Informationen von einem eigenen Webserver abruft. Bisher fand diese Kommunikation mit dem Internet über eine unverschlüsselte Verbindung mittels einer sogenannten Ethercard, einer Ethernet Erweiterung für den Arduino, statt.

Zum Testen der zahlreichen Angriffsmethoden auf SSL musste also zuerst eine verschlüsselte Verbindung hergestellt werden, um im Anschluss versuchen zu können diese zu knacken. Da der Arduino selbst nicht genug Rechenleistung zum herstellen einer SSL Verbindung besitzt musste das Setup entsprechend verändert werden. Dazu wurde die Ethercard durch einen WiFi-Chip des Typs ATWINC1500 ersetzt, welcher die benötigte SSL Funktionalität mit sich bringt. 

Nach Umrüstung der Hardware musste die Software entsprechend angepasst werden, um vom neuen Netzwerkinterface Gebrauch zu machen. Mit der Arduino WiFi101-Bibliothek \cite{arduino:WiFi} wird eine SSL Verbindung zum Webserver aufgebaut. Der Webserver selbst war bereits für SSL eingerichtet, womit nur Änderungen beim Arduino-Client nötig waren. Die SSL Zertifikate werden manuell mittels eines Firmware-Updaters auf das WiFi-Modul geladen werden.
\begin{lstlisting}[label={lst:port443},caption={Port 443 ohne Fallback auf Port 80}]
if (client.connect(server, 443)) {
...
}
\end{lstlisting}
Nach Implementierung aller Änderungen wurden diverse Angriffsversuche unternommen. Das blockieren von Port 443, um eine SSL Verbindung zu unterbinden und einen Fallback auf Port 80 zu erzwingen, erwies sich wie vorhergesehen als erfolglos, da schlicht kein Fallback im Code \ref{lst:port443} vorgesehen ist und der Arduino ausschliesslich auf Port 443 einen Verbindungsaufbau versucht.
Auch die nächste Methode SSL-Strip blieb erfolglos aufgrund der expliziten Nutzung von SSL ohne Fallback.
Zuletzt wurde mittels SSL-Sniffing probiert ob gefälschte Zertifikate vom WiFi-Chip akzeptiert werden um die Verschlüsselung zu umgehen, jedoch erwies sich auch diese Methode als nutzlos, da nur die korrekt eingespielten SSL Zertifikate akzeptiert werden. 

Die Matrix ist somit gegen allen getesteten Angriffsmethoden immun, aufgrund von korrekter Konfiguration und strikter Verwendung von SSL. 
\section{Quellenverzeichnis}
\printbibliography[heading=none]
\end{document}
